% # INFO:
% this file will produce a few warnings, even when working correctly:
% - using fallback Bibtex backend, cased because of the older backend, which may not support all features
% - Bad type area setting, caused by using a landscape page. The type area is malformed, but this is no problem if you only include external pages


% # choosing a document class:
% The command below is the first one you will see in pretty much any LaTeX document. It sets up the so called documentclass,
% which is what determines all of your styling and sets up loads of stuff for you. In general there are classes for pretty much
% every type of document you could need, but this project provides a few classes specialized for the Hochschule Hannover.
% You can find information about the classes and the options in the README, but here is an example setup with load of comments:
\documentclass[	%----------------------Preamble---------------------------------------------------%
		fontsize=11pt,  % fontsize
		a4paper,	    % papersize
		%twoside,		% double sided layout
		german,		% document language (also numberingsystem)
		%ngerman,		% document language (also numberingsystem)
		sans,			% font type (sans/roman)
		f4,				% HsH facultie (f1-f5)
		%draft			% quicker compilations, images are not included
	]{HsH-report}		% documentclass


% # the preamble:
% everything between `\docuemtnclass' and `\begin{document}' is called the preamble. Here you configure all settings for your document.
% The `\documentclass' command is actually part of that configuration. Lets see what you could do here:


% # packages
% To extend LaTeXes basic functionality, you can load additional packages. Some are already loaded, but others you load in as needed.
% Check the README about what's already loaded and what's recommended.
% Also, this classfiles automaticly check for a file called `config.tex', in which you can collect common configurations and reuse them across
% files in this and/or multiple projects. If you use the full template from GitLab, it provied a `config.tex' which only applies the settings,
% if the corresponding package is loaded, making it even more reusable.
%
% Here are some packages we are going to need with some explanations:
\usepackage{color}		% for colouring stuff
\usepackage{siunitx}	% units
\usepackage{listings}	% including formated code snippets
\usepackage{csvsimple}	% for importing CSV files
\usepackage{subfigure}	% for subfigures
\usepackage{soul}		% strikethrough text
\usepackage{amssymb}	% for spectial Math symbols
\usepackage{enumitem}	% more list options
\usepackage{lipsum}		% dummy text


% # bibliography
% While you can just create a super simple bibliography directly in your documents and format it completely yourself (see <https://en.wikibooks.org/wiki/LaTeX/Bibliography_Management#Embedded_system>)
% it is far more manageable and consistent to use a system called BibTeX, which allows you to maintain a file of all your sources and takes care of all
% the formatting for you. It also figures out which sources you use in your documents and only prints these, allowing for a single bibliography file
% you can use on multiple/all your projects. You can also create bibliographies on a chapter basis, if you prefer.
% To use it, just load the according package in your preamble, as shown below.
% Be aware that you need to run a separate program (`bibtex' or`biber') on your latex file for your citations to be rendered. But you usually don't
% need to run that every time.
% Also while this example is set up to use `bibtex' (because it is the default in loats of editors), the defaul for this project is the more modern
% `biber' command, so you need to change your editor accordingly if you omit the `backend=bibtex' option below:
\usepackage[backend=bibtex]{biblatex}

% now we load our bibliography file. Open it to see what it looks like
\addbibresource{bibliography.bib}


% # document information
% In your preamble you also list your documents information and metadata. These will be used on the title-page as well as being available throughout
% the document. Additionally, these documentclasses set up the resulting PDF file with the appropriate Metadata.
% You can just delety any of this comands or leave them empty if you don't need it for a project.
% See the following examples and what they create in the PDF file:
\author{
	Paul Depping,
	Moritz Möller
} % the author and matrikelnr commands could also be on a single line, this is just more readable
\matrikelnr{
	1715018,
	1714792
}
\titlehead{BIN-Seminar}
\subject{Strukturiertes Testen}
\title{Statische und dynamische Testmethoden}
\subtitle{Übersicht über verschiedene Testmethoden}
\date{\today}
\professor{Denis Beßen}

% Now that you are all set, let's begin with the actual content of the document.
% Don't forget the corresponding `\end{document}'!
\begin{document} %----------- beginning of document -----------------------------------------------

% for longer documents it is custom to have different numbering until the first page of actuall content.
% For that use this command to switch to Roman pagenumbers and turn off chapternumbers:
\frontmatter

% While you can of course create your own title-page, either with latex or externally, the easiest way is to use the build in command.
% These classes redefine it to include the HsH-logo (depending on the chosen faculty) and to use the additional data provided in the preamble.
% You can also use the optional argument to change the title-pages alignment to l,c or r:
\maketitle[c]

% this command is provided by these documentclasses. It creates a standard Text at the bottom of the page and a line to sign on for every author.
% You are not restricted to this exact position and can use it where ever you want in your document, if you prefer it at the back, but it there.
\declarationAuthorship

% # abstract
% sometimes you are required to also create an abstract. Use this environment for that.
% It will create a new page and a heading for you as well as indenting the whole text block a little.
% if you have provided keywords, they will also be put at the end of the abstract.
\begin{abstract}
	Testmethoden lassen sich in zwei große Gruppen unterteilen:
	Statische Tests, welche ohne Ausführung des Programmcodes auskommen, und dynamische Tests, welche den Programmcode ausführen.
	In dieser Ausarbeitung wird Paul Depping auf statische Testmethoden und Max Möller auf dynamische Testmethoden eingehen
	und einen Überblick über die verschiedenen Unterkategorien von Tests schaffen.
\end{abstract}

% This command will create the table of contents (TOC).
% Keep in mind, that LaTeX can only now about things it has already read. For everything that follows, it relies on temporary files which are created
% on the fly. So the complete TOC will only be rendered after at leas two LaTeX runs.
\tableofcontents

% the following command is the counterpiece of the `\frontmatter' command.
% It resets pagenumbers so that the next chapter is the first with actuall content.
\mainmatter


% now we can begin with the actuall relevant content. So let's beginn by creating the first chapter:
\chapter{Statische Tests -- Paul Depping} \label{chap: static}
	\section{Strukturierte Gruppenprüfungen}
		Beispielzitat \cite{fagan:advances}
		\subsection{Grundlagen}
			A \cite{dpunkt:basiswissensoftwaretest}.
		\subsection{Reviews}
			B \cite{fagan:design}
			\lipsum[1]
		\subsection{Grundlegende Vorgehensweise}
			Hi \cite{freedman:reviews}
			\lipsum[1]
		\subsection{Rollen und Verantwortlichkeiten}
			Testing \cite{hanser:qualität}
			\lipsum[1]
		\subsection{Reviewarten}

	\section{Statische Analyse}
		\lipsum[1-2]
		\subsection{Compiler als statisches Analysewerkzeug}
			\lipsum[1]
		\subsection{Prüfung der Einhaltung von Konventionen und Standards}
			\lipsum[1]
		\subsection{Durchführung der Datenflussanalyse}
			\lipsum[1]
		\subsection{Durchführung der Kontrollflussanalyse}
			\lipsum[1]
		\subsection{Ermittlung von Metriken}
			\lipsum[1]
	\section{Korrektheitsbeweise}
		\lipsum[1-2]
	\section{Symbolische Programmausführung}
		\lipsum[1-2]

\chapter{Dynamische Tests -- Moritz Möller} \label{chap: dynamic}
	\lipsum[1-20]

\printbibliography
\end{document}
